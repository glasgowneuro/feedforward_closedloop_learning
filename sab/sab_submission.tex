\documentclass{llncs}

\usepackage{graphicx}                                        % for pdf, jpeg, png and tif graphics
\usepackage{amsmath}                                         % for align
\usepackage{amssymb}                                         % for >= and <= signs
\usepackage{xfrac}                                           % for nice x/y fractions (sfrac)
\usepackage{tikz}
\usepackage{amsbsy}                     % for boldsymbol
\usepackage{upgreek}

\mathchardef\mhyp="2D    % define math mode hyphen

\allowdisplaybreaks[4]

\title{Deep feedback learning}

\author{Bernd Porr \and Paul Miller}

\institute{Glasgow Neuro, bernd,paul@glasgowneuro.tech}

\begin{document}

\maketitle

\begin{abstract}
\end{abstract}

\section{Introduction}

\section{Deep feedback learning}
We define a multi layered network where every neuron is a standard computational
unit which calculates the weighted sums of its inputs and then sent through an
activation function:
\begin{equation}
  v_j = \Theta\left( \sum_i w_{ij} v_{i} \right)
\end{equation}
where the activity feeds from neurons in layer $v_i$ to neurons in layer $v_j$
and so forth. The activities are weighted by the weights $w_{ij}$.

The errors are also injected into the input layer and then propagrated through
the network:
\begin{equation}
  e_j = \frac{\left( \sum_i w_{ij} e_{i} \right) \Theta^\prime (v_j) }{\sqrt{\sum_i w_{ij}}}
\end{equation}
where the $\Theta^\prime (v_j)$ is the derivative from the activation function.

The weight change is then calculated in a heterosynaptic fashion:
\begin{equation}
  w_{ij} = w_{ij} + \gamma v_i * e_j
\end{equation}
where $\gamma$ is the learning rate and the weights are normalised

Learning is then performed in two passes: first the activity is propagated and
then the error signal is propagated via the same mechanism.

\section{Results}

\section{Discussion}

\bibliographystyle{splncs03}

\bibliography{sab_submission,isab,ours}

\end{document}



\begin{figure}[h!]
  \centering
  \includegraphics[width=\columnwidth]{arena_cct.pdf}
  \caption{A. Overview of the simulation environment. The arena has two markers, labelled R and B (red and blue), within   \label{fig:cct}
    }
\end{figure}

